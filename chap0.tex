
\pagenumbering{arabic}
\setcounter{page}{1}

\chapter{Introductory Remarks}

We're living in an era that internet is one of the daily needs of human life and modern countries, instead of going to banks people use internet banking and instead of sending a physical letter they use digital ways of communication. This leads to a more digital robust way of living, but it should be considered that this means people are trusting the middle companies and third parties for their online services. The most important one are banks and financial middle mans (\eg credit card companies) and there has been many downsides to these trust, such as banks failing\footnote{List of bank failures in the United States (2008–present) \url{https://en.wikipedia.org/wiki/List_of_bank_failures_in_the_United_States_(2008–present)}}, government collapses that leads to the countries currency exchange rate fall to pennies (\eg Zimbabwean dollar ~\cite{hanke2009measurement} ) and many more examples in smaller scales. The need of having a digital form of money that is not being controlled by one entity is obvious.

Bitcoin is the first decentralized virtual currency and by far has adopted the most number of users ~\cite{Nak08}. it is based on cryptographic functions to remove the need of a central bank and regulates the generation of new units. Bitcoin is still on its early stages and there has not been that many practical usage of this digital currency that could offer an ultimate solution for financial problems.

In this thesis, we would like to look at available tools for end user to hold and use bitcoin with a usability and security perspective, and then evaluate the possibilities for a small business to be able to accept Bitcoin payments. This could summaries the usage of any kind of a currency as there only should be two entities involved in a monetary transaction, the payee and the payer.


\textbf{Thesis Statement:} End-to-end usable payment system using Bitcoin, and its components, can be designed for real-world deployability while maintaining a strong notion of usability and security.

\section{Background}
The concept of digital cash was introduced by David Chaum in 1983 ~\cite{chaum1983blind}, he continued the idea and founded a company named DigiCash\footnote{\url{https://en.wikipedia.org/wiki/DigiCash}} as a digital cash company but filed for bankruptcy in 1998. In the same year, Paypal\footnote{\url{http://paypal.com}} emerged and other systems such as E-gold \footnote{\url{https://en.wikipedia.org/wiki/E-gold}} followed but due to criminal usage of E-gold they got shut down in 2005 by United States Federal. In 2008, Bitcoin was introduced and solved most of the problems of all the digital cashes before it which marked the start of digital currencies.

Even though the idea of having a decentralize money is interesting, there has not been any work on the  usability of this new form of money.

\fixme{what else to write here!!?}

\fixme{cognitive walkthrough?}

\fixme{requirement engineering?}


\section{Contributions}
While this research is one of the first usability researches on the subject of Bitcoin, our work provides a number of new contributions toward the evaluation of Bitcoin wallet clients and payment systems, we summarize them here:

\textbf{Bitcoin Wallet comparison framework:} we come up with a framework for comparing Bitcoin wallet clients and evaluate the existing tools.

\textbf{Bitcoin Point of Sale comparison framework: } with the focus on the available tools for businesses to accept bitcoin as a method of payment, we analyses a small business's requirements using SCRAM a requirement engineering method and evaluate all the available options for bitcoin payments.

\textbf{Fully customizable open-source Bitcoin Point of Sale\footnote{\url{https://github.com/shayanb/Bitcoin-PoS-PHP}} }: later we develop a customized Bitcoin point of sale specific to a small business needs as none of the available approaches could satisfy the needs. This software is available under GNU General Public Licience v2 and has already been used in other small businesses to accept bitcoin.

Our focus in this thesis, is mostly on usability of these tools and having a framework for comparing and evaluating future tools.

\section{Organization}
In the next chapter (Chapter 2), we present some background on Bitcoin, the underlying protocol as long as it fits the scale of this thesis, and also some details about the methods used in the next chapters such as Cognitive walkthrough and requirement engineering.

On Chapter 3, we use cognitive walkthrough to evaluate the usability of bitcoin wallet clients and then develop a framework for comparing existing and future wallet clients. This published work was the first ever usability paper on the Bitcoin subject, mostly focused on how these clients handle key management that is the fundamental requirement in any Bitcoin wallet client.

In chapter 4, we survey all the available tools for small businesses to accept bitcoins and evaluate them based on the framework introduced in the same chapter. Then using a requrement engineering method we list all the advantages and disadvates of using each method for a small business and later we develop a fully customized open-srouce Bitcoin point of sale and implement it in real-world cafe to accept bitcoins. This cafe\footnote{ Cafe Aunja \url{http://aunja.com}}. is the first cafe in eastern Canada that accepts Bitcoin.

\fixme{maybe something about last chapter conclusions ? }
