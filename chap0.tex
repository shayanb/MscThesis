


\chapter{Introductory Remarks}

\pagenumbering{arabic}
\setcounter{page}{1}

We live in an era where Internet is one of the daily needs of human life and modern countries. Instead of going to banks, people use Internet banking and instead of sending a physical letter they use digital ways of communication. This leads to a robust digital way of living, but this also means people are trusting middle companies and third parties for their online services. The most important ones are banks and financial middle man (\eg credit card companies) and there has been many downsides to the trust, such as banks failing\footnote{List of bank failures in the United States (2008–present) \url{https://en.wikipedia.org/wiki/List_of_bank_failures_in_the_United_States_(2008–present)}}, government collapses that leads to the country's currency exchange rate decrease to pennies (\eg Zimbabwean dollar ~\cite{hanke2009measurement} ) and many more examples on smaller scale. The need of having a digital form of money that is not being controlled by one entity is plain to see.

Bitcoin is the first decentralized virtual currency and by far has the most number of users ~\cite{Nak08}. It is based on cryptographic functions to remove the need of a central bank and regulates the generation of new units. Bitcoin is still in its early stages and there have not been that many practical applications of this digital currency that could offer an ultimate solution for financial problems.

In this thesis, we would like to look at available tools to facilitate users in holding and using Bitcoin by a perspective on usability and security, and then evaluate the possibilities for a small business to be able to accept Bitcoin payments. This could be a summary for the usage of any kind of a currency, as there only should be two entities involved in a monetary transaction, the payee and the payer.\\

\textbf{Thesis Statement:} End-to-end usable payment systems using Bitcoin, and its components, can be designed for real-world deployability while maintaining a strong notion of usability and security.

\section{Background}
The concept of digital cash was introduced by David Chaum in 1983 ~\cite{chaum1983blind}, he continued the idea and founded a company named DigiCash\footnote{\url{https://en.wikipedia.org/wiki/DigiCash}} as a digital cash company. DigiCash filed for bankruptcy in 1998 and sold its assets to eCash Technologies, another digital currency company. In the same year, Paypal\footnote{\url{http://paypal.com}} emerged and other systems such as E-gold \footnote{\url{https://en.wikipedia.org/wiki/E-gold}} followed but due to unregulated use of E-gold they got shut down in 2005 by the federal government of the United States. In 2008, Bitcoin was introduced and solved most of the problems of all the digital cash before it which marked the start of digital currencies.

Bitcoin was an innovation because it was not a new digital version of cash (eCash) nor a commodity like gold (e-Gold), but it was its own currency with properties that were not seen in any other currencies before.

Even though the idea of having a decentral money is interesting, to the best of our knowledge, there has not been any published work on the usability of this new form of money. Thus a framework to be able to evaluate applications in this field with usability perspective is needed.

For Bitcoin to flourish, adoption must expand beyond developers and tech-savvy enthusiasts to novice users. Expansion solidifies the need for a usable, comprehensible approach to Bitcoin. If users cannot safely manage Bitcoin keys, it may result in the users' loss of funds and/or a poor reputation for Bitcoin, both of which could dissuade further user adoption. 


%\fixme{what else to write here!!?}

%\fixme{cognitive walkthrough?}

%\fixme{requirement engineering?}


\section{Contributions}
While this research is one of the first usability research on the subject of Bitcoin, our work provides a number of new contributions toward the evaluation of Bitcoin wallet clients and payment systems, we summarize them here:

\textbf{Bitcoin wallet comparison framework:} We design a framework for comparing Bitcoin wallet clients and evaluate the existing tools.

\textbf{Bitcoin point of sale comparison framework: } with the focus on the available tools for businesses to accept Bitcoin as a method of payment, we analyze a small business's requirements using SCRAM~\cite{REScenario} a requirement engineering method ~\cite{dorfman1990system} and evaluate all the available options for Bitcoin payments.

\textbf{Fully customizable open-source Bitcoin point of sale\footnote{\url{https://github.com/shayanb/Bitcoin-PoS-PHP}} }: later we develop a customized Bitcoin point of sale specific to a small business needs as none of the available approaches could satisfy the needs. This software is available under GNU General Public Licience v2 and has already been used in other small businesses to accept Bitcoin.

Our focus in this thesis is on usability of these tools and having a framework for comparing and evaluating future tools. This is one of the first researches in this field and would be considered as a novel work on usability of Bitcoin.

\section{Organization}
In the next chapter (Chapter 2), we present some background on Bitcoin, the underlying protocol to the extent that it fits the scale of this thesis, and also some details about the methods used in the next chapters such as cognitive walkthrough~\cite{WRLP94}  and requirement engineering.

In Chapter 3, we use cognitive walkthrough to evaluate the usability of Bitcoin wallet clients and then develop a framework for comparing existing and future wallet clients. This work is the first published usability paper on Bitcoin subject, mostly focused on how these clients handle key management that is the fundamental requirement of any Bitcoin wallet client.

In chapter 4, we survey all the available tools for small businesses to accept Bitcoins and evaluate them based on the framework introduced in the same chapter. Then using a requirement engineering method we list all the advantages and disadvantages of using each method for a small business and later we develop a fully customized open-source Bitcoin point of sale and implement it in real-world cafe to accept Bitcoins. This cafe\footnote{ Cafe Aunja \url{http://aunja.com}}. is the first cafe in Quebec, Canada that accepts Bitcoin.

And in the end we evaluate our contributions and discuss the future work needed in this field to have a more usable and robust system for holding and accepting Bitcoin as a method of payment.